 %\documentclass[aps,twocolumn,secnumarabic,balancelastpage,amsmath,amssymb,nofootinbib, floatfix]{revtex4}
\documentclass[aps,twocolumn,secnumarabic,balancelastpage,amsmath,amssymb,nofootinbib,floatfix]{revtex4-1}


%%%%%%%%%%%%%%%%%%%%%%%%%%%%%%%%%%%%%%%%%%%%%%%%%%%%%%%%%%%%%%%%%%%


\usepackage{graphicx}      % tools for importing graphics
%\usepackage{lgrind}        % convert program code listings to a form
                            % includable in a LaTeX document
%\usepackage{xcolor}        % produces boxes or entire pages with
                            % colored backgrounds
%\usepackage{longtable}     % helps with long table options
%\usepackage{epsf}          % old package handles encapsulated postscript issues
\usepackage{bm}            % special bold-math package. usge: \bm{mathsymbol}
\usepackage{subcaption}
%\usepackage{asymptote}     % For typesetting of mathematical illustrations
%\usepackage{thumbpdf}
\usepackage[colorlinks=true]{hyperref}  % this package should be added after
                                        % all others.
                                        % usage: \url{http://web.mit.edu/8.13}
\usepackage{wasysym}

\begin{document}
\title{July 2022 Summary of Project to Determine Asteroid Shape and Density from Flybys}
\author{Jack Dinsmore, Julien de Wit}
\email{jtdinsmo@mit.edu}
\date{\today}
\affiliation{MIT Department of Physics}

\newcommand{\abs}[1]{\left| #1 \right|}
\newcommand{\parens}[1]{\left( #1 \right)}
\newcommand{\brackets}[1]{\left[ #1 \right]}
\newcommand{\comment}[1]{\textcolor{red}{\emph{ #1 }}}
\newcommand{\x}{\bm{\hat x}}
\newcommand{\y}{\bm{\hat y}}
\newcommand{\z}{\bm{\hat z}}
\newcommand{\J}{\mathcal{J}}
\newcommand{\M}{\mathcal{M}}
\newcommand{\R}{\mathcal{R}}





\begin{abstract}
    The purpose of this summary is to derive the math governing the way an asteroid reacts to a non-point source. It was motivated by the discovery that, under the assumption of a gravitating point source and a small asteroid, the tidal torque can be written in terms of components of the moment of inertia, which means inference is limited to the six components of the inertia matrix. By violating these assumptions and considering other effects (oblateness, etc.), we can break the degeneracy.
\end{abstract}

\maketitle


%%%%%%%%%%%%%%%%%%%%%%%%%%%%%%%%%%%%%%%%%%%%%%%%%%%%%%%%%%%%%%%%%%

\section{Definitions}
Please see \href{https://citeseerx.ist.psu.edu/viewdoc/download?doi=10.1.1.56.5257&rep=rep1&type=pdf}{info on spherical harmonics}
$$P_{lm}=\frac{1}{2^ll!}(1-t^2)^{m/2} \frac{d^{l+m}}{dl^{l+m}}(t^2-1)^l$$
$$Y_{lm}(\theta, \phi) = P_{lm}(\cos\theta)e^{im\phi}$$
$$\bar Y_{lm}(\theta, \phi) = (-1)^m \sqrt{\frac{2l+1}{4\pi}\frac{(l-m)!}{(l+m)!}} Y_{lm}(\theta, \phi)$$
$$R_{lm}(\bm r) = (-1)^m \frac{r^l}{(l+m)!}Y_{lm}(\bm r)$$
$$S_{lm}(\bm r) = (-1)^m \frac{(l-m)!}{r^{l+1}}Y_{lm}(\bm r)$$
We write $Y(\bm r)=Y(\hat{\bm r})$.

These definitions lead to several key facts. For $r' < r$,
$$\frac{1}{|\bm r - \bm r'|}=\sum_{lm}R_{lm}(\bm r')S^*_{lm}(\bm r).$$
We also have a translation rule:
$$S_{lm}=\sum_{l'm'}R^*_{l'm'}(\bm r')S_{l+l',m+m'}(\bm r)$$
which I have used to show, via Mathematica, for $D > r$,
$$S_{lm}^*(\bm r - \bm D)=(-1)^l (-1)^m \sum_{l'=|m|}^\infty R_{l'm}^*(\bm r) \frac{(l+l')!}{D^{l+l'+1}}.$$
Similarly, we have some gradient formulas that result in
$$\bm r \times \nabla R_{lm}^*=\frac{1}{2}[(\x + i\y)(l-m+1)R^*_{l,m-1}$$
$$+(-\x+i\y)(l+m+1)R^*_{l,m+1}+2im\z R^*_{lm}]$$
\comment{This last one is wrong; the way I have it in writing is correct. But for some reason, the mistake cancels, and the final answer is correct anyway.}
All of these functions obey the symmetry relations $F_{l,-m}=(-1)^mF_{lm}^*$
\comment{WRONG! Y DOES NOT! FIX THIS!!!}.


\section{Derivation}
Consider an asteroid at distance $D$ from a gravitating body. Place the origin near the asteroid (it will be convenient later for the origin to be at the center of mass of the asteroid, but not necessary), and orient the $\z$ axis to point towards the gravitating body. For brevity, we will call the gravitating body a planet, but it could be a star or another asteroid. No assumptions are made about the nature of the body except that its distance to the asteroid can be placed inside a sphere which does not contain the asteroid.

Consider the gravitational field $V$ of the body, defined by
$$V = -G\int_M d^3 r' \rho_M(\bm r')\frac{1}{|\bm r' - \bm r|}.$$
The force dealt on an object of mass $m$ by such a potential is $\bm F = -m\nabla V$. This potential can be expanded into spherical harmonics by one of the defintions:
$$V = -G\sum_{lm}S^*_{lm}(\bm r)\int_M d^3 r' \rho_M(\bm r')R_{lm}(\bm r').$$
For convenience, we define
$$\J_{lm} = \int_M d^3 r' \rho_M(\bm r')R_{lm}(\bm r')$$
so that the potential and force experience by small amount of mass $dm$, is
$$V = -G\sum_{lm}S^*_{lm}(\bm r)\J_{lm},\indent \frac{d\bm F}{dm}=G\sum_{lm}\J_{lm} \nabla_{\bm r} S^*_{lm}(\bm r).$$
Consequentially, the torque on an entire asteroid is
$$\bm \tau = G\sum_{lm}\J_{lm} \int d^3 r\rho(\bm r)\bm r \times \nabla_{\bm R} S^*_{lm}(\bm R)$$
where the integral is carried out over the smaller body, and $\bm R$ represents the vector pointing from the center of the planet to the point of integration, which is $\bm R = \bm r - \bm D$ if $\bm D$ points to the center of the asteroid.

The word ``center'' here is loosely defined because the center of the planet need only be the point around which $\J_{lm}$ is calculated, and the center of the asteroid need only be the origin of $\bm r$. They do not need to be the center of mass, although it may be computationally easier. The centers do need to be close to their respective bodies in some sense, or else the assumptions about $D > r$ will be violated.

We will start by translating $S^*_{lm}(\bm R)$. Via an equation in the definition,
\begin{equation*}
\begin{aligned}
\bm \tau = &G\sum_{lm}\J_{lm} \int d^3 r\rho(\bm r)\bm r \times \nabla_{\bm R}\\
&\brackets{(-1)^l (-1)^m \sum_{l'=|m|}^\infty R_{l'm}^*(\bm r) \frac{(l+l')!}{D^{l+l'+1}}}.
\end{aligned}
\end{equation*}
Since $\nabla_{\bm R}=\nabla_{\bm r}$, the equation is separable into an integral which is computed locally:
\begin{equation*}
\begin{aligned}
\bm \tau = &(-1)^l (-1)^m G\sum_{lm}  \frac{\J_{lm}}{D^{l+1}}\sum_{l'=|m|}^\infty \frac{(l+l')!}{D^{l'}}\\
& \int d^3 r\rho(\bm r)\bm r\times \nabla_{\bm r}R_{l'm}^*(\bm r).
\end{aligned}
\end{equation*}

Note for ease of unit analysis that the integral is unitless.

Now we will compute the gradient of the equation by one of the equations in the preamble:
\begin{equation*}
\begin{aligned}
\bm \tau = &(-1)^l (-1)^m G\sum_{lm}  \frac{\J_{lm}}{2D^{l+1}}\sum_{l'=|m|}^\infty \frac{(l+l')!}{D^{l'}}\\
& \int d^3 r\rho(\bm r)[(\x + i\y)(l'-m+1)R^*_{l',m-1}(\bm r)\\
&+(-\x+i\y)(l'+m+1)R^*_{l',m+1}(\bm r)+2im\z R^*_{l'm}(\bm r)]
\end{aligned}
\end{equation*}
and we had better hope that this is real. To proove that it is real, we combine $m$ and $-m$ terms in the first sum, restricting ourselves to the off-diagonal ($m \neq 0$) terms. We will do the diagonal terms later.
\begin{equation*}
\begin{aligned}
\bm \tau_o = &G\sum_{l=1}^\infty \sum_{m=1}^l \frac{(-1)^l}{2D^{l+1}}\sum_{l'=m}^\infty \frac{(l+l')!}{D^{l'}}\Bigg\{(-1)^m \J_{lm}\\
& \int d^3 r\rho(\bm r)\Big[(\x + i\y)(l'-m+1)R^*_{l',m-1}(\bm r)\\
&+(-\x+i\y)(l'+m+1)R^*_{l',m+1}(\bm r)+2im\z R^*_{l'm}(\bm r)\Big] \\
&+ (-1)^{-m} \J_{l,-m}\\
&\int d^3 r\rho(\bm r)\Big[(\x + i\y)(l'+m+1)R^*_{l',-m-1}(\bm r)\\
&+(-\x+i\y)(l'-m+1)R^*_{l',-m+1}(\bm r)-2im\z R^*_{l',-m}(\bm r)\Big]\Bigg\}
\end{aligned}
\end{equation*}
We make use of the relation that $(-1)^m R_{l,-m}=R^*_{lm}$ to get
\begin{equation*}
\begin{aligned}
\bm \tau_o = &G\sum_{l=1}^\infty \sum_{m=1}^l \frac{(-1)^l}{2D^{l+1}}\sum_{l'=m}^\infty \frac{(l+l')!}{D^{l'}}\Bigg\{(-1)^m \J_{lm}\\
& \int d^3 r\rho(\bm r)\Big[(\x + i\y)(l'-m+1)R^*_{l',m-1}(\bm r)\\
&+(-\x+i\y)(l'+m+1)R^*_{l',m+1}(\bm r)+2im\z R^*_{l'm}(\bm r)\Big] \\
&+ (-1)^{-m} \J_{l,-m}\\
&\int d^3 r\rho(\bm r)\Big[-(\x + i\y)(l'+m+1)R_{l',m+1}(\bm r)\\
&-(-\x+i\y)(l'-m+1)R_{l',m-1}(\bm r)-2im\z R_{l',m}(\bm r)\Big]\Bigg\}
\end{aligned}
\end{equation*}

Now let us compute the relationship between $\J_{lm}$ and $\J_{-lm}$. From the definition of $\J_{lm}$,
$$\J_{l,-m} = \int_M d^3 r' \rho_M(\bm r')R_{l,-m}(\bm r')$$
$$\J_{l,-m} = (-1)^m\int_M d^3 r' \rho_M(\bm r')R^*_{l,m}(\bm r')$$
$$\J_{l,-m} = (-1)^m\J_{lm}^*$$
so it obeys the same relation. Thus,
\begin{equation*}
\begin{aligned}
\bm \tau_o = &G\sum_{l=1}^\infty \sum_{m=1}^l \frac{(-1)^l(-1)^{-m}}{2D^{l+1}}\sum_{l'=m}^\infty \frac{(l+l')!}{D^{l'}}\\
&\Bigg\{\J_{lm} \int d^3 r\rho(\bm r)\Big[(\x + i\y)(l'-m+1)R^*_{l',m-1}(\bm r)\\
&+(-\x+i\y)(l'+m+1)R^*_{l',m+1}(\bm r)+2im\z R^*_{l'm}(\bm r)\Big] \\
&+ \J_{l,m}^* \int d^3 r\rho(\bm r)\Big[-(\x + i\y)(l'+m+1)R_{l',m+1}(\bm r)\\
&-(-\x+i\y)(l'-m+1)R_{l',m-1}(\bm r)-2im\z R_{l',m}(\bm r)\Big]\Bigg\}
\end{aligned}
\end{equation*}
which is, in turn
\begin{equation*}
\begin{aligned}
\bm \tau_o = &G\sum_{l=1}^\infty \sum_{m=1}^l \frac{(-1)^l(-1)^{-m}}{2D^{l+1}}\sum_{l'=m}^\infty \frac{(l+l')!}{D^{l'}}\\
&\Bigg\{\int d^3 r\rho(\bm r)\Big[(\x + i\y)(l'-m+1)\J_{lm} R^*_{l',m-1}(\bm r)\\
&+(-\x+i\y)(l'+m+1)\J_{lm} R^*_{l',m+1}(\bm r)+2im\z \J_{lm} R^*_{l'm}(\bm r) \\
&-(\x + i\y)(l'+m+1)\J_{l,m}^* R_{l',m+1}(\bm r)\\
&-(-\x+i\y)(l'-m+1)\J_{l,m}^* R_{l',m-1}(\bm r)-2im\z \J_{l,m}^* R_{l',m}(\bm r)\Big]\Bigg\}
\end{aligned}
\end{equation*}
and finally
\begin{equation*}
\begin{aligned}
\bm \tau_o = &G\sum_{l=1}^\infty \sum_{m=1}^l \frac{(-1)^l(-1)^{-m}}{2D^{l+1}}\sum_{l'=m}^\infty \frac{(l+l')!}{D^{l'}}\\
&\Bigg\{\int d^3 r\rho(\bm r)\Big[\\
&(l'-m+1)\Big[(\x + i\y)\J_{lm} R^*_{l',m-1}(\bm r)\\
&-(-\x+i\y)\J_{l,m}^* R_{l',m-1}(\bm r)\Big]\\
&+(l'+m+1)\Big[(-\x+i\y)\J_{lm} R^*_{l',m+1}(\bm r)\\
&- (\x + i\y)\J_{l,m}^* R_{l',m+1}(\bm r)\Big]\\
&+2im\z\Big[\J_{lm} R^*_{l'm}(\bm r)-\J_{l,m}^* R_{l',m}(\bm r)\Big]\Big]\Bigg\}.
\end{aligned}
\end{equation*}
Just changing some signs for convenience,
\begin{equation*}
\begin{aligned}
\bm \tau_o = &G\sum_{l=1}^\infty \sum_{m=1}^l \frac{(-1)^l(-1)^{-m}}{2D^{l+1}}\sum_{l'=m}^\infty \frac{(l+l')!}{D^{l'}}\\
&\Bigg\{\int d^3 r\rho(\bm r)\Big[\\
&(l'-m+1)\Big[(\x + i\y)\J_{lm} R^*_{l',m-1}(\bm r)\\
&+(\x-i\y)\J_{l,m}^* R_{l',m-1}(\bm r)\Big]\\
&+(l'+m+1)\Big[(-\x+i\y)\J_{lm} R^*_{l',m+1}(\bm r)\\
&+(-\x - i\y)\J_{l,m}^* R_{l',m+1}(\bm r)\Big]\\
&+2im\z\Big[\J_{lm} R^*_{l'm}(\bm r)-\J_{l,m}^* R_{l',m}(\bm r)\Big]\Big]\Bigg\}.
\end{aligned}
\end{equation*}

Since $2\Re z = z + z^*$ and $2\Im z = i(z^* - z)$, we may write
\begin{equation*}
\begin{aligned}
\bm \tau_o = &G\sum_{l=1}^\infty \sum_{m=1}^l \frac{(-1)^l(-1)^{-m}}{D^{l+1}}\sum_{l'=m}^\infty \frac{(l+l')!}{D^{l'}}\\
&\Bigg\{\int d^3 r\rho(\bm r)\Big[\\
&(l'-m+1)\Re\Big[(\x + i\y)\J_{lm} R^*_{l',m-1}(\bm r)\Big]\\
&+(l'+m+1)\Re\Big[(-\x+i\y)\J_{lm} R^*_{l',m+1}(\bm r)\Big]\\
&+2m\z\Im\Big[\J_{l,m}^* R_{l',m}(\bm r)\Big]\Big]\Bigg\}.
\end{aligned}
\end{equation*}
Now we have proved that the off diagonal terms of torque are real. To make it even more agnostic of imaginary numbers, another way to put the above equation is that
\begin{equation*}
\begin{aligned}
\bm \tau_o = &G\sum_{l=1}^\infty \sum_{m=1}^l \frac{(-1)^l(-1)^{-m}}{D^{l+1}}\sum_{l'=m}^\infty \frac{(l+l')!}{D^{l'}}\Bigg\{\int d^3 r\rho(\bm r)\Big[\\
&\x\bigg((l'-m+1)\Re\Big[\J^*_{lm} R_{l',m-1}(\bm r)\Big]\\
&-(l'+m+1)\Re\Big[\J^*_{lm} R_{l',m+1}(\bm r)\Big]\bigg)\\
&+\y\bigg((l'-m+1)\Im\Big[\J^*_{lm} R_{l',m-1}(\bm r)\Big]\\
&+(l'+m+1)\Im\Big[\J^*_{lm} R_{l',m+1}(\bm r)\Big]\bigg)\\
&+2m\z\Im\Big[\J_{l,m}^* R_{l',m}(\bm r)\Big]\Big]\Bigg\}.
\end{aligned}
\end{equation*}

The on-diagonal ($m=0$) terms of torque are as follows:
\begin{equation*}
\begin{aligned}
\bm \tau_d = &(-1)^l G\sum_{l=0}^\infty  \frac{\J_{l0}}{2D^{l+1}}\sum_{l'=0}^\infty \frac{(l+l')!}{D^{l'}}\\
& \int d^3 r\rho(\bm r)[(\x + i\y)(l'+1)R^*_{l',-1}(\bm r)\\
&+(-\x+i\y)(l'+1)R^*_{l',1}(\bm r)].
\end{aligned}
\end{equation*}
Already we see that the $l'=0$ term cannot come into play, so we skip it. Then, making use of the complex conjugate identity,
\begin{equation*}
\begin{aligned}
\bm \tau_d = &-(-1)^l G\sum_{l=0}^\infty  \frac{\J_{l0}}{2D^{l+1}}\sum_{l'=1}^\infty \frac{(l+l')!(l'+1)}{D^{l'}}\\
& \int d^3 r\rho(\bm r)[(\x + i\y)R_{l',1}(\bm r)+(\x-i\y)R^*_{l',1}(\bm r)].
\end{aligned}
\end{equation*}
The result is simple enough:
\begin{equation*}
\begin{aligned}
\bm \tau_d = &-(-1)^l G\sum_{l=0}^\infty  \frac{\J_{l0}}{D^{l+1}}\sum_{l'=1}^\infty \frac{(l+l')!(l'+1)}{D^{l'}}\\
& \int d^3 r\rho(\bm r)\Re[(\x + i\y)R_{l',1}(\bm r)]
\end{aligned}
\end{equation*}
or
\begin{equation*}
\begin{aligned}
\bm \tau_d = &-G\sum_{l=0}^\infty  \frac{(-1)^l \J_{l0}}{D^{l+1}}\sum_{l'=1}^\infty \frac{(l+l')!(l'+1)}{D^{l'}}\\
& \int d^3 r\rho(\bm r)\parens{\x\Re [R_{l',1}(\bm r)] - \y\Im [R_{l',1}(\bm r)]}.
\end{aligned}
\end{equation*}
This is also manifestly real since $\J_{l0}$ is always real.

Suppose we have the following precomputed matrix:
$$\M_{lm} = \int d^3 r \rho(\bm r) R_{lm} \bm (r),$$
which is an identical definition to $\J$, except for the asteroid rather than for the planet. Then we have
\begin{equation*}
\begin{aligned}
\bm \tau_o = &G\sum_{l=1}^\infty \sum_{m=1}^l \frac{(-1)^l(-1)^{-m}}{D^{l+1}}\sum_{l'=m}^\infty \frac{(l+l')!}{D^{l'}}\Bigg\{\\
&\x\bigg((l'-m+1)\Re\Big[\J^*_{lm} \M_{l',m-1}\Big]\\
&-(l'+m+1)\Re\Big[\J^*_{lm} \M_{l',m+1}\Big]\bigg)\\
&+\y\bigg((l'-m+1)\Im\Big[\J^*_{lm} \M_{l',m-1}\Big]\\
&+(l'+m+1)\Im\Big[\J^*_{lm} \M_{l',m+1}\Big]\bigg)\\
&+2m\z\Im\Big[\J_{l,m}^* \M_{l',m}\Big]\Bigg\}
\end{aligned}
\end{equation*}
\begin{equation*}
\begin{aligned}
\bm \tau_d = &G\sum_{l=0}^\infty  \frac{(-1)^l \J_{l0}}{D^{l+1}}\sum_{l'=1}^\infty \frac{(l+l')!(l'+1)}{D^{l'}}\\
& \parens{-\x\Re [\M_{l',1}] + \y\Im [\M_{l',1}]}.
\end{aligned}
\end{equation*}
Finally,
$$\bm \tau = \bm\tau_d + \bm\tau_o.$$
Note that
$$\Re[\J_{l,-m}^* M_{l',-m-1}]=-\Re[\J_{lm} M^*_{l',m+1}]=-\Re[\J^*_{lm} M_{l',m+1}]$$
and
$$\Im[\J_{l,-m}^* M_{l',-m-1}]=-\Im[\J_{lm} M^*_{l',m+1}]=\Im[\J^*_{lm} M_{l',m+1}].$$
Therefore, we may write
\begin{equation*}
\begin{aligned}
\bm \tau_o = &G\sum_{l=1}^\infty \sum_{|m|\neq 0}^l \frac{(-1)^l(-1)^{-m}}{D^{l+1}}\sum_{l'=|m|}^\infty \frac{(l+l')!}{D^{l'}}\Bigg\{\\
&\x(l'-m+1)\Re\Big[\J^*_{lm} \M_{l',m-1}\Big]\\
&+\y(l'-m+1)\Im\Big[\J^*_{lm} \M_{l',m-1}\Big]\\
&+2m\z\Im\Big[\J_{l,m}^* \M_{l',m}\Big]\Bigg\}.
\end{aligned}
\end{equation*}
Note that $\bm \tau_d$ is actually the $m=0$ case of this sum, Thus,
\begin{equation*}
\begin{aligned}
\bm \tau = &G\sum_{l=0}^\infty \sum_{m=-l}^l \frac{(-1)^l(-1)^{m}}{D^{l+1}}\sum_{l'=|m|}^\infty \frac{(l+l')!}{D^{l'}}\Bigg\{\\
&\x(l'-m+1)\Re\Big[\J^*_{lm} \M_{l',m-1}\Big]\\
&+\y(l'-m+1)\Im\Big[\J^*_{lm} \M_{l',m-1}\Big]\\
&+2m\z\Im\Big[\J_{l,m}^* \M_{l',m}\Big]\Bigg\}.
\end{aligned}
\end{equation*}

\section{Rotation}
The constants $\M$ rotate over the course of the asteroid's orbit due to the asteroid's own rotation. Let the constants $\M$ be calculated at some time $t_0$ in the orbit, and define the rotation $\R(t)$ such for any vector $\bm r$ pointing to some point in the asteroid at time $t_0$, $\bm r' = \R(t) \bm r$ points to the same point at time $t$. Let $\M'$ be the constants calculated at time $t$. Then
$$\M'_{lm} = \int' d^3 r' \rho'(\bm r') R_{lm}(\bm r'),$$
$$\M'_{lm} = \int d^3 r \rho(\bm r) R_{lm}(\R(t)\bm r).$$
Now we use the following fact:
$$\overline{Y_{lm}} = \sum_{m'=-l}^l\parens{D^l_{mm'}(\R(t))}^*\overline{Y_{lm'}}(\bm r),$$
where $D^l_{mm'}$ are the components of the \href{https://en.wikipedia.org/wiki/Wigner_D-matrix#Relation_to_spherical_harmonics_and_Legendre_polynomials}{Wigner-$D$ matrix}, and $\overline{Y_{lm}}$ are the normalized spherical harmonics. For our purposes, this amounts to
$$Y_{lm}(\bm r') = \sum_{m'=-l}^l\sqrt{\frac{(l+m)!}{(l-m)!}\frac{(l-m')!}{(l+m')!}}\parens{D^l_{mm'}(\R(t))}^*Y_{lm'}(\bm r),$$
or
\begin{equation*}
\begin{aligned}
R_{lm}(\bm r') = &\sum_{m'=-l}^l\sqrt{\frac{(l-m')!}{(l-m)!}\frac{(l+m')!}{(l+m)!}}\\
&\parens{D^l_{mm'}(t)}^*R_{lm'}(\bm r)(-1)^{m+m'},
\end{aligned}
\end{equation*}
where we have written $D^l_{mm'}(t)=D^l_{mm'}(\R(t))$. Finally,
\begin{equation*}
\begin{aligned}
\M'_{lm} =& \int d^3 r \rho(\bm r) \\
&\sum_{m'=-l}^l\sqrt{\frac{(l-m')!}{(l-m)!}\frac{(l+m')!}{(l+m)!}}\\
&\parens{D^l_{mm'}(t)}^*R_{lm'}(\bm r)(-1)^{m+m'}\\
=&\sum_{m'=-l}^l\sqrt{\frac{(l-m')!}{(l-m)!}\frac{(l+m')!}{(l+m)!}}(-1)^{m+m'}\\
&\parens{D^l_{mm'}(t)}^*\M_{lm'}.
\end{aligned}
\end{equation*}

The problem of computing the tidal torque on an asteroid over an orbit is then reduced to computing the constants $M_{lm}$ at some point in the orbit, then computing the components of $D^l_{mm'}$ at all other points in the orbit.

\section{Asteroid model that is friendly to tidal torque computation}
Assume

We want to design an asteroid model with adjustable density profile and shape, which is also easily able to calculate $\M_{lm}$. One such asteroid is as follows:
Define the density at point $\bm r$ to be
$$\rho(\bm r) = \sum_{l=0}^L \sum_{m=-l}^l \sum_{s=0}^\infty C_{lms}Y^*_{lm}(\hat{\bm r})r^s$$
with parameters $C_{lms} \in \mathbb{C}$. The constants $\M_{lm}$ are as follows:
\begin{equation*}
\begin{aligned}\M_{lm} = &\sum_{l'=0}^\infty \sum_{m'=-l'}^{l'} \sum_{s=0}^\infty\int d^3 r  C_{l'm's}Y^*_{l'm'}(\hat{\bm r})\\
&r^s (-1)^m \frac{r^l}{(l+m)!}Y_{lm}(\bm r).
\end{aligned}
\end{equation*}
\comment{Actually, this radial distribution is a terrible idea. It diverges for $s=0$ and isn't natural. Maybe just fit $\M_{lm}$?}
This can be broken into spherical coordinates and integrated, taking advantage of the fact that
$$\int d\Omega Y_{lm}Y^*_{l'm'}=\delta_{ll'}\delta_{mm'}\frac{4\pi}{2l+1}\frac{(l+m)!}{(l-m)!},$$
so
\begin{equation*}
\begin{aligned}\M_{lm} = &\sum_{l'=0}^\infty \sum_{m'=-l'}^{l'} \sum_{s=0}^\infty\int_0^R r^2 dr  C_{l'm's}\\
&r^s \frac{r^l}{(l-m)!}\delta_{ll'}\delta_{mm'}\frac{4\pi}{2l+1}(-1)^m.
\end{aligned}
\end{equation*}
This, in turn, becomes
$$\M_{lm} = \frac{4\pi (-1)^m}{(2l+1)(l-m)!}\sum_{s=0}^\infty C_{lms} \frac{R^{l+s+3}}{l+s+3}.$$
We want to require that the density be real, which we do by combining the $m$ and $-m$ terms:
$$\rho(\bm r) =  \sum_{s=0}^\infty r^s\sum_{l=0}^L C_{l0s}Y_{l0} + \sum_{m=1}^l \parens{C_{lms}Y_{lm} + C_{l,-m,s}Y_{l,-m}}$$

$$\rho(\bm r) =  \sum_{s=0}^\infty r^s\sum_{l=0}^L C_{l0s}Y_{l0} + \sum_{m=1}^l \parens{C_{lms}Y_{lm} + (-1)^m C_{l,-m,s}Y^*_{lm}},$$
and therefore if $C_{lms} = (-1)^m C_{l,-m,s}^*$, our density is real.

The reader may think it useful to use a different radial function instead of $r^s$, and indeed other functions might provide better fits. One choice that seems most preferable would be radial functions $B_{n}(r)$ that are orthonormal to $r^n$, so that each $\M_{lm}$ can be written with no sum. Unfortunately such functions do not exist. One can get close with
$$B_n(x)=\sum_{k=n}^K 2^k \binom{n}{k} \binom{\frac{n+k-1}{2}}{n}P_k(x)\frac{2n + 1}{2}$$
where $P_k(x)$ are the Legendre polynomials, and with $K \in \mathbb{N}$. Here, $\int_{-1}^1 B_n r^m dx = \delta_{nm}$ for all $n, m < K$, but this sum diverges for large $K$. An alternative, derived via Taylor series, is to define $B_n = \sum_k a_{kn}x^k$, where $a_{kn}$ are the components of the matrix $A=C^{-1}$, and the components of matrix $C$ are $c_{mk}=\frac{1}{m+k+1}$. This sum also diverges.


\section{Moment of inertia}
The moment of inertia matrix of an object is written as
$$I_{xx}=\int d^3 r \rho(\bm r) (y^2 + z^2), \indent \dots$$
$$I_{xy}=-\int d^3 r \rho(\bm r) xy, \indent \dots.$$
Note that
$$x^2=r^2\brackets{-\frac{1}{3}(Y_{20}-1)+2Y_{2,-2}+\frac{1}{12}Y_{22}}$$
$$y^2=r^2\brackets{-\frac{1}{3}(Y_{20}-1)-2Y_{2,-2}-\frac{1}{12}Y_{22}}$$
$$z^2=r^2\brackets{\frac{2}{3}Y_{20}+\frac{1}{3}}$$
$$xz=r^2\brackets{\frac{Y_{21}}{6}-Y_{2,-1}}$$
$$yz=r^2\brackets{\frac{Y_{21}}{6i}+\frac{Y_{2,-1}}{i}}$$
$$xy=r^2\brackets{\frac{Y_{22}}{12i}-2\frac{Y_{2,-2}}{i}}.$$
Then we can write the above in terms of $R_{2m}$:
$$I_{xx}=\int d^3r \rho(\bm r)-\frac{1}{3}(2R_{20}-1)+2R_{2,-2}+2R_{22}$$
$$I_{yy}=\int d^3r \rho(\bm r)-\frac{1}{3}(2R_{20}-1)-2R_{2,-2}-2R_{22}$$
$$I_{zz}=\int d^3r \rho(\bm r)\frac{4}{3}R_{20}+\frac{1}{3}$$
$$I_{xz}=-\int d^3r \rho(\bm r)R_{21}-R_{2,-1}$$
$$I_{yz}=-\int d^3r \rho(\bm r)\frac{R_{21}+R_{2,-1}}{i}$$
$$I_{xy}=-\int d^3r \rho(\bm r)2\frac{R_{22}-R_{2,-2}}{i}$$
and therefore
$$I_{xx}=-\frac{1}{3}(2\M_{20}-1)+2\M_{2,-2}+2\M_{22}$$
$$I_{yy}=-\frac{1}{3}(2\M_{20}-1)-2\M_{2,-2}-2\M_{22}$$
$$I_{zz}=\frac{4}{3}\M_{20}+\frac{1}{3}$$
$$I_{xz}=-\M_{21}+\M_{2,-1}$$
$$I_{yz}=-\frac{\M_{21}+\M_{2,-1}}{i}$$
$$I_{xy}=-2\frac{\M_{22}-\M_{2,-2}}{i}.$$

Thus, the moment of inertia is determined entirely by $\M_{lm}$ and therefore by $C_{lms}$. The equation of motion of the asteroid,
$$\dot{\bm \omega} = I^{-1}(\bm \tau - \bm\omega \times (I \bm\omega)),$$
is now entirely known as a function of orientation.



\section{Degeneracy}
We are interested in looking at the degeneracy of the $C_{lms}$ parameters with respect to their effect on the equation of motion. First of all, note that $l'=0$ gives no torque. The $\M_{00}$, or spherically symmetric term, will then never come into play. Also, if we take the origin of $\M$ to be the asteroid's center of mass, all of the first order moments $\M_{1,m}$ will vanish as well, meaning that the dynamics of the asteroid depend only on $C_{lms}$, $l\geq 2$. This incidentally requires that the torque term that is lowest order in $1/D$ is $(l,l',m)=(0,2,0)$, and is proportional to $1/D^3.$ It evaluates to
$$\tau \approx \frac{3G\J_{00}}{D^3}(\x I_{xz}-\y I_{yz}).$$
\comment{This indicates a mistake!}. Since $Y_{00}=1$, we know $\J_{00}$ is equal to the mass of the planet. So this agrees with the result achieved from a Taylor series expansion of tidal force, assuming a point mass for the Earth.

This relationship between first-order torque and the inertia tensor indicates that $\M_{lm}$ may be orthogonal in some sense; identifying one does not affect the rest. We may therefore hope to fit $C_{l'm's}$ independently of $C_{lms}$. We also expect $\J_{00}\gg \J_{lm}$, where $l\neq 0$, and we expect $D$ to be large. Consequentially, $C_{lms}$ will be damped by a factor of $\frac{l-2}{D}$ compared to the first order coefficients $C_{2ms}$, which determine $\M_{2m}$ and therefore the moment of inertia matrix. This is good news; it means that our $\M_{lm}$ parameterization is well-aligned with the system.

Unfortunately, the $s$ component of the $C_{lms}$ parameterization is not well-aligned. It is not obvious which terms will dominate, and as a result they may have a great deal of uncertainty, which could negatively affect the uncertainty of the entire fit. I think it's best to just fit $\M_{lm}$ instead.


\clearpage


\section{Matrix representation}
We can rearrange the sum, since
$$\sum_{l=1}^\infty \sum_{m=-l}^l\sum_{l'=|m|}^\infty = \sum_{l=1}^\infty \sum_{l'=0}^\infty \sum_{m=-\text{min}(l,l')}^{\text{min}(l,l')}.$$

Let us define $\tau = \x \Re\tau_{\parallel} + \y \Im \tau_{\parallel} + \hat z \Im\tau_\perp$. We wish to determine $\tau_\parallel$ and $\tau_\perp$ by matrix multiplication.

Suppose we define $\mathcal{P}_{l'm} = (-1)^{-m}(l' - m + 1) \M_{l',m-1}$ and
$\mathcal{L}_{l'm}=(-1)^{-m}m\M_{l'p}$. Then
$$\tau_\perp =G\sum \sum \sum \frac{(-1)^l}{D^{l+1}} \frac{(l+l')!}{D^{l'}}\J_{lm}^*\mathcal{P}_{l'm}$$
$$\tau_\parallel =G\sum \sum \sum \frac{(-1)^l}{D^{l+1}} \frac{(l+l')!}{D^{l'}}\J_{lm}^*\mathcal{L}_{l'm}.$$
We can write the above as a matrix multiplication by defining the infinite-dimensional matrix
$$\J_{ab} = \begin{cases}
    \J_{l=a,m=b-a} & b \leq 2a+1 \\
    0 & \text{else}
\end{cases}$$
and similarly for $\mathcal{P}$ and $\mathcal{L}$.


\bibliography{../asteroid-flybys.bib}
\end{document}
